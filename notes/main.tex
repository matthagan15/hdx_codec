\documentclass{article}
\usepackage[utf8]{inputenc}

\usepackage{amsmath,amsthm, amssymb}
\usepackage[margin=3cm]{geometry}
\usepackage{mathtools}
\usepackage{dsfont}
\usepackage{xcolor}
\usepackage{algorithm,algpseudocode}
\usepackage{todonotes}
\usepackage{nicefrac}
\usepackage{mathrsfs}
\usepackage{tikz}
\usepackage{thm-restate}


%%%%%%%%    THEOREM DEFINITIONS AND RESTATABLE
% \newcounter{claim}
% \setcounter{claim}{0}
\newtheorem{theorem}{Theorem}
\newtheorem{lemma}[theorem]{Lemma}
\newtheorem{corollary}[theorem]{Corollary}
\newtheorem{claim}[theorem]{Claim}

\usepackage{todonotes}

\newcommand{\matt}[1]{\todo[color=red!50, prepend, caption={Matt}, tickmarkheight=0.25cm]{#1}}
\newcommand{\note}[1]{\emph{Note: #1}}
\newcommand{\conjecture}[1]{ \noindent\emph{\textbf{Conjecture:}} \emph{ #1 }}




%%%%%%%%    NOTATION DEFINITIONS FOR EASIER WRITING
\newcommand{\ket}[1]{|#1\rangle}
\newcommand{\bra}[1]{\langle #1|}
\newcommand{\braket}[2]{\langle #1|#2\rangle}
\newcommand{\ketbra}[2]{| #1\rangle\! \langle #2|}
\newcommand{\parens}[1]{\left( #1 \right)}
\newcommand{\brackets}[1]{\left[ #1 \right]}
\newcommand{\abs}[1]{\left| #1 \right|}
\newcommand{\norm}[1]{\left| \left| #1 \right| \right|}
\newcommand{\diamondnorm}[1]{\left| \left| #1 \right| \right|_\diamond}
\newcommand{\anglebrackets}[1]{\left< #1 \right>}
\newcommand{\overlap}[2]{\anglebrackets{#1 , #2 }}
\newcommand{\set}[1]{\left\{ #1 \right\}}
\newcommand{\ceil}[1]{\left\lceil #1 \right\rceil}
\newcommand{\openone}{\mathds{1}}
\newcommand{\expect}[1]{\mathbb{E}\brackets{#1}}
\newcommand{\variance}[1]{\textit{Var} \brackets{ #1 }}
\newcommand{\prob}[1]{\text{Pr}\left[ #1 \right]}
\newcommand{\bigo}[1]{O\left( #1 \right)}
\newcommand{\bigotilde}[1]{\widetilde{O} \left( #1 \right)}
\newcommand{\ts}{\textsuperscript}
\newcommand{\field}{\mathbb{F}}

\DeclareMathOperator{\Tr}{Tr}
\newcommand{\trace}[1]{\Tr \brackets{ #1 }}
\newcommand{\partrace}[2]{\Tr_{#1} \brackets{ #2 }}
\newcommand{\complex}{\mathbb{C}}

%%%%% COMMONLY USED OBJECTS
\newcommand{\hilb}{\mathcal{H}}
\newcommand{\partfun}{\mathcal{Z}}
\newcommand{\identity}{\mathds{1}}
\newcommand{\gue}{\rm GUE}
\DeclareMathOperator{\sinc}{sinc}
\DeclareMathOperator{\im}{im}
\DeclareMathOperator{\hermMathOp}{Herm}
\DeclareMathOperator{\lc}{lc}
\newcommand{\herm}[1]{\hermMathOp\parens{#1}}


\title{Notes on HDX Codes}
\author{Matthew Hagan}
\date{January 17, 2024}

\begin{document}

\maketitle
These notes are intended to serve as a workspace for me to record or work thoughts on the High Dimensional Expander (HDX) codes by Dinur, Liu, Zhang. As I am working on an implementation of an encoder and decoder for these codes that I am writing from scratch there are a lot of pieces that I need to implement and I want to have a good grasp of how each of these pieces works individually and as part of the whole code. 

\section{Polynomials over a Finite Field}
One of the main types the codec uses is polynomials over a finite field with a single indeterminate, denoted $R = \field_p [x]$. We will typically use $p$ as a prime, not a prime power. This set is a ring, as we can add, subtract, and associatively multiply two polynomials $f, g \in \field_p[x]$. Moreover, $R$ is equipped with a valuation function $\nu : R \ \set{0} \to \mathbb{N}_{\geq 0}$. For simplicity, we can define $\nu(0) = - 1$, all we need is that the valuation of 0 is less than the valuation of any nonzero element of $R$. 

This valuation is Euclidean, meaning that for all $f \in R$ and $q \in R$, with $g \neq 0$, there exists polynomials $g, r \in R$ such that $f = q * g + r$ and either $r = 0$ or $\nu(r) < \nu(g)$. For this case, the function $\nu = \deg$ works and also gives us the nice properties that $\nu(a * b) = \nu(a) + \nu(b)$ and $\nu(a + b) = \max(\nu(a), \nu(b))$. There is some kind of tropical algebra going on here? This is screaming division with $q$ as the quotient and $r$ as the remainder. Given an $f$ and $q$, we are going to iteratively pick polynomials $s_i$ such that $f = s_i* q_i + r_i$ is maintained at each step. By varying the degree of $q_i$ we can control the degree of $r_i$ through the properties of the valuation $\nu = \deg$. We will work through the first few steps of the algorithm below and afterwards state it more formally. 

Note as $\nu(r) < \nu(g) \leq \nu(g * q) = \nu(g) + \nu(q)$ and $\nu(a + b) = \max (\nu(a), \nu(b))$ we have that 
$$\nu(f) = \nu(q * g + r) = \max(\nu(g * q), \nu(r)) = \max(\nu(g) + \nu(q), \nu(r)) = \nu(g) + \nu(q).$$
Given $f, g$ we can find $q,r$ by constructing a sequence of polynomials $q_i, r_i$ such that $f = q_i * g + r_i$ for all $i$. 

We note the special case in which $\deg(f) < \deg(g)$, in which case we have that $q = 0$ necessarily. To prove this we work by contradiction, so we assume $q \neq 0$ and show that this implies $\deg(f) \geq \deg(g)$. We first show that $\deg(q) \geq 1$ leads to a contradiction. If $\deg(q) \geq 1$, then we have that $\deg(q * g) = \deg(q) + \deg(g) \geq \deg(g) + 1$. However by the above, we know that $\deg(f) = \deg(g) + \deg(q) \geq \deg(g) + 1 \geq \deg(g)$. This is a contradiction that $\deg(f) < \deg(g)$. The only remaining cases are $q = c \in \field_p \set{0}$, in which case $\deg(q) = 0$, or $q = 0$ and $\deg(q) = -1$. If $q = c \in \field_p$, then $\deg(r) < \deg(q) = 0 \implies r = 0$ and $\deg(f) = \deg(g) + \deg(q) = \deg(g)$. This also violates our assumption. The only remaining case is $\deg(q) = -1$ and $q = 0$, implying $f = r$. 

From the above argument, we can therefore restrict our attention to the case in which $\deg(f) \geq \deg(g)$. For purely illustrative purposes of the algorithm, lets consider the case in which $\deg(f) = \deg(g)$. For this we write out $f$ and $g$ in terms of their coefficients, and let $d = \deg(f) = \deg(g)$. 
\begin{align}
    f(x) &= f_d x^d  + f_{d-1} x^{d-1} + f_1 x^1 + \ldots f_0 \\
    g(x) &= g_d x^d  + g_{d-1} g^{d-1} + g_1 x^1 + \ldots g_0.
\end{align}
Now we consider a nonzero polynomial $q$. As $\deg(f) = \deg(g) + \deg(q)$ and  $\deg(f) = \deg(g)$ in our case, we infer $\deg(q) = 0$ and $q \in \field_p$ is a constant. The product 
\begin{equation}
    q(x) * g(x) = \sum_{k = 0}^{d + n} \sum_{i + j = k} q_i g_j x^{i + j},
\end{equation}
is therefore simplified to $\sum_{k = 0}^{d} q * g_k x^{k}$. We note that setting $q * g_d = f_d$, or $q = g_d^{-1} f_d$ with $g_d^{-1} * g_d = 1$ (as guaranteed by the properties of a field), lets us guarantee that $r = f - q * g$ can be computed as the subtraction
\begin{align}
    r(x) &= f(x) - q * g(x) \\
    &= \sum_{k = 0}^{d} f_k x^k - \sum_{k' = 0}^d (g_d^{-1} f_d) g_{k'} x^{k'} \\
    &= f_d x^d - (g_d^{-1} f_d) g_d x^d + \sum_{k = 0}^{d - 1} f_k x^k - \sum_{k' = 0}^{d - 1} q * g_{k'} x^{k'} \\
    &= \sum_{k = 0}^{d - 1} (f_k - (f_d g_d^{-1} ) g_k) x^k, \label{eq:equal_degree_remainder}
\end{align}
and we see that $\deg(r) = d - 1 < d = \deg(g)$ as required. So if you are dividing $f(x)$ by a polynomial $g(x)$ with $\deg(f) = \deg(g)$, then the quotient is $q = f_d * g_d^{-1}$ and the remainder is given by Eq. \eqref{eq:equal_degree_remainder}. To simplify notation in the future we will let $\lc$ denote the leading coefficient of a polynomial, $\lc(\sum_{k=0}^d a_k x^k) = a_d$.

To consider the most general case it is helpful to play around with the usage of $g$ and $q$ to get different remainders. Assume that $\deg(f) = \deg(g) + 1$ and that therefore $q = q_1 x^1 + q_0$. Multiplying $q$ and $g$ leads to 
\begin{align}
    q(x) * g(x) &= q_1 x^1 g(x) + q_0 g(x) \\
    &\eqqcolon q_1 * t(x) + q_0 g(x),
\end{align}
where we have defined $t(x) = x^1 * g(x)$. We now have simplified our problem of division if we just look at the process of dividing $f$ by $t(x)$. We know that $\deg(t) = \deg(g) + 1 = \deg(f)$. From the case we considered in detail, we know that the promised quotient is $q' = \lc(f) * \lc(g)^{-1}$. The remainder we can then read off,
\begin{equation}
    r' = f(x) - q' t(x).
\end{equation} 

Now that we know $q_1$, we return to the original polynomial
\begin{equation}
    f(x) = q_0 g(x) + q_1 x^1 g(x) + r.
\end{equation}
We now note that the subtraction $f(x) - q_1 x^1 g(x)$ kills of the leading term of $f$. So we have an $f' = f(x) - q_1 x^1 g(x)$ that has $\deg(f') = \deg(f) - 1 = \deg(g)$. We have
\begin{equation}
    f'(x) = q_0 g(x) + r = q'' g(x) + r''.
\end{equation}
as $\deg(f') = \deg(g)$, we can solve for $q'' = \lc(f') * \lc(g)^{-1} = f_{d} * \lc(g)^{-1}$. however, $q'' = q_0$, so we have now found the coefficients needed for the original $q$. We can then read off $r = f(x) - q(x) g(x)$, with the guarantee that $\deg(r) < \deg(g)$ or $r = 0$. 

We now have to extend this argument to an inductive claim.


\section{Coset Complex}
We now introduce Coset Complexes, the objects used to construct high dimensional expanders. These have nice properties when used as the backbone for an error correcting code. The code will be constructed from the cosets formed by subgroups of a group, so we start with the elements of the group. The group will be a multiplicative group of matrices with entries over a ring $R_n$. The ring will be the quotient ring $\frac{\mathbb{F}_q[t]}{\langle \phi(t) \rangle}$, where $\phi \in \mathbb{F}_q[t]$ is an irreducible and primitive polynomial in $\mathbb{F}_q [t]$ of degree $n$ and $\mathbb{F} = \mathbb{F}_q$ is a fixed finite field. For technical purposes we will need to choose the degree $n$ such that $3 \nmid q^n - 1$, which occurs if $q \not\equiv 1 \mod 3$. 

For a correctly chosen $\phi$, the group we will work with is $G = SL_3(R_n)$ and is best described in terms of it's generators. The fact that we are using 3x3 matrices will determine the dimensionality of the resulting complex, we will have triangles as our highest dimension objects. The subgroups we will use are given by
\begin{align}
    K_1 &= \set{\begin{bmatrix}1 & a t & c t^2 \\ 0 & 1 & b t \\ 0 & 0 & 1 \end{bmatrix} \in M_3(R_n)} \\
    K_2 &= \set{\begin{bmatrix}1 & 0 & 0 \\ c t^2 & 1 & a t \\ b t & 0 & 1 \end{bmatrix} \in M_3(R_n)} \\
    K_3 &= \set{\begin{bmatrix}1 & b t & 0 \\ 0 & 1 & 0 \\ at & ct^2 & 1 \end{bmatrix} \in M_3(R_n)}.
\end{align}
Or we could use an alternative set of generators
\begin{align}
    H_1 &= \set{ \begin{bmatrix} 1 & a t & 0 \\ 0 & 1 & 0 \\ 0 & 0 & 1 \end{bmatrix} : a \in \mathbb{F}} \\
    H_2 &= \set{\begin{bmatrix} 1 & 0 & 0 \\ 0 & 1 & at \\ 0 & 0 & 1 \end{bmatrix}: a \in \mathbb{F}} \\
    H_3 &= \set{ \begin{bmatrix} 1 & 0 & 0 \\ 0 & 1 & 0 \\ at & 0 & 1 \end{bmatrix} : a \in \mathbb{F}}.
\end{align}
This makes it apparent that we are considering elementary matrix operations. If a matrix from $H_1$ acts on another from the right ($M * X_1$ with $X_1 \in H_1$) then it is adding $at$ times the first column of $M$ to the second column of $M$. The overall group $G$ is then generated by $\langle K_1, K_2, K_3 \rangle = \langle H_1, H_2, H_3 \rangle$. 

We now show how to construct the simplicial complex that we will define our codes over. We start with the base set $X(0)$, or the nodes of the complex, aka the vertices, aka the zero-dimensional faces. For each element $g \in G$ we construct three vertices, the coset $g K_1 = \set {g k_1 : \forall k_1 \in K_1}$, the coset $g K_2$, and the coset $g K_3$. Each vertex then clearly has a type $\tau(v) \in \set{1, 2, 3}$. The set of vertices is then the disjoint union of all of these cosets. We construct an edge between two vertices if the cosets intersect. We see right away that we do not have any edges between the same type of vertices,
\begin{align}
    g K_1 \cap h K_1 \implies g k_1 = h k_2 \implies g = h k_2 k_1^{-1} = h k \implies g \in h K_1 \implies g K_1 = h K_1.
\end{align}

Once we have constructed all the edges we can now compute the triangles in the complex. We say a triangle $\set{g_1 K_1, g_2 K_2, g_3 K_3}$ is in the complex if the intersection $g_1 K_1 \cap g_2 K_2 \cap g_3 K_3 \neq \varnothing$ is not empty. 


\section{Quantum Codes}
We assume familiarity with the basics of quantum mechanics necessary for quantum error correction, namely quantum states (both density matrix and state vector formalisms), quantum channels, and Pauli matrices. We review the stabilizer formalism briefly before moving to CSS codes.




\end{document}